\documentclass[10pt,a4paper,twoside,titlepage]{article}
%\usepackage[utf8]{inputenc}
\usepackage[ngerman]{babel}
\usepackage{fontspec}
\usepackage{amsmath}
\usepackage{amsfonts}
\usepackage{amssymb}
\usepackage{graphicx}
\usepackage{leading}

\setmainfont{Palatino Linotype}

\author{Ivanildo Kowsoleea}
\title{Wie visualisiert man einen Hyperwürfel}

\newcommand{\myeq}[1]{
	\begin{equation}
		\begin{split}
			#1
		\end{split}
	\end{equation}
}

\begin{document}
\maketitle
\leading{13pt}
\parindent{2pt}

\section{Was ist ein Hyperwürfel?}
Ein Hyperwürfel ist ein Würfel in der vierten Dimension. Wie sieht so
ein Ding aus, und wie konstruiert man so etwas?

Um die Konstruktion besser zu verstehen fangen wir mit einem 0-dimensionalen
Würfel an: einen Punkt.




\end{document}