\documentclass[10pt,a4paper,twoside,titlepage]{article}
%\usepackage[utf8]{inputenc}
\usepackage[ngerman]{babel}
\usepackage{fontspec}
\usepackage[hidelinks]{hyperref}
\usepackage{amsmath}
\usepackage{amsfonts}
\usepackage{amssymb}
\usepackage{graphicx}
\usepackage{leading}
\usepackage{parskip}

\setmainfont{Palatino Linotype}

\author{Ivanildo Kowsoleea}
\title{Wie visualisiert man einen Hyperwürfel}

\newcommand{\myeq}[2]{
	\begin{equation}
		\begin{split}
			#1
		\end{split}
		\label{#2}
	\end{equation}
}

% makes an image
% arg1: filename of image
% arg2: caption
% arg3: label (for \autoref{} )
% arg4: width in cm
\newcommand{\image}[4]{
	\begin{figure}[!ht]
		\centering
		\includegraphics[width=#4cm]{#1}
		\caption{#2}
		\label{#3}
	\end{figure}
}

\begin{document}
\maketitle
\leading{13pt}

\section{Was ist ein Hyperwürfel?}
Ein Hyperwürfel ist ein Würfel in der vierten Dimension. Wie sieht so
ein Ding aus, und wie konstruiert man so etwas?

Um die Konstruktion besser zu verstehen fangen wir mit einem 0-dimen\-siona\-len
Würfel an: einen Punkt. Da wir hier nur einen Punkt zur Verfügung 
braucht man auch keine Koordinaten. Beim Übergang zum eindimensionalen Fall 
\textemdash\ ein Linienstück \textemdash\ nehmen wir den ursprünglichen Punkt
und vergeben die eindimensionale Koordinate $(0)$. Wir Kopieren diesen Punkt,
vergeben hier die Koordinate $(1)$, und verbinden beide Punkte mit einander.
Das jetzt entstandene Linienstück ist unser eindimensionaler Würfel.

\myeq{(0)\rightarrow (1)}{dim1}

Wir breiten jetzt aus zur zweiten Dimension. Das Linienstück in (1)
bekommt eine zweite Koordinate gleich 0:
\myeq{(0,0)\rightarrow (1,0)}{dim1in2}
Anschließend kopieren wir dieses Linienstück und geben der Kopie als zweite
Koordinate die 1. Danach verbinden wir jeden Punkt mit seiner Kopie.

\myeq{(0,0)\rightarrow (1,0)\\
	(0,1)\rightarrow(1,1)\\
	(0,0)\rightarrow(0,1)\\
	(1,0)\rightarrow(1,1)}{dim2}
	
Diese vier Linienstücke bilden ein Quadrat \textemdash\ die zweidimensionale
Variante des Würfels (siehe \autoref{quadrat} auf \autopageref{quadrat}).

\image{img/quadrat.eps}{Ein Quadrat}{quadrat}{5}

Die vier Koordinatenpaare in (3) geben genau die Liniensegmente an, die
mit einander verbunden werden müssen, um ein Quadrat zu erzeugen.

Genau so finden wir die Liniensegmente die wir für ein Würfel (dreidimensional)
brauchen. Wir nehme die 4 Liniensegemnte aus (3) und versehen sie mit einer
dritten Koordinate die wir gleich 0 stellen. Danach nehmen wir wieder die 
Liniensegmente, diesmal aber mit der 1 als dritte Koordinate. Die 
vier Eckpunkte verbinden wir noch mit deren Kopien.

\myeq{(0,0,0)\rightarrow(1,0,0)\\
	(0,1,0)\rightarrow(1,1,0)\\
	(0,0,0)\rightarrow(0,1,0)\\
	(0,1,0)\rightarrow(1,1,0)\\
	\\
	(0,0,1)\rightarrow(1,0,1)\\
	(0,1,1)\rightarrow(1,1,1)\\
	(0,0,1)\rightarrow(0,1,1)\\
	(0,1,1)\rightarrow(1,1,1)\\
	\\
	(0,0,0)\rightarrow(0,0,1)\\
	(0,1,0)\rightarrow(0,1,1)\\
	(1,0,0)\rightarrow(1,0,1)\\
	(1,1,0)\rightarrow(1,1,1)
}{würfel}

\image{img/wuerfel.eps}{Ein Würfel}{würfel1}{6}

Die 12 Liniensegmente wür den Würfel sind gegeben in (4). \autoref{würfel1} auf
\autopageref{würfel1} zeigt den Würfel nochmal.

Die Konstruktion des Hyperwürfels stellt jetzt kein Problem mehr da. Wir
versehen die 12 Liniensegmente aus (4) mit einer 4. Koordinate gleich 0.
Das wiederholen wir mit einer 4. Koordinate gleich 1 und verbinden die 8
Eckpunkte mit deren Kopien. Das Ergebnis ist:

\myeq{(0,0,0,0)\rightarrow(1,0,0,0);\
	(0,1,0,0)\rightarrow(1,1,0,0)\\
	(0,0,0,0)\rightarrow(0,1,0,0);\
	(0,1,0,0)\rightarrow(1,1,0,0)\\
	(0,0,1,0)\rightarrow(1,0,1,0);\
	(0,1,1,0)\rightarrow(1,1,1,0)\\
	(0,0,1,0)\rightarrow(0,1,1,0);\
	(0,1,1,0)\rightarrow(1,1,1,0)\\
	(0,0,0,0)\rightarrow(0,0,1,0);\
	(0,1,0,0)\rightarrow(0,1,1,0)\\
	(1,0,0,0)\rightarrow(1,0,1,0);\
	(1,1,0,0)\rightarrow(1,1,1,0)\\
	\\
	(0,0,0,1)\rightarrow(1,0,0,1);\
	(0,1,0,1)\rightarrow(1,1,0,1)\\
	(0,0,0,1)\rightarrow(0,1,0,1);\
	(0,1,0,1)\rightarrow(1,1,0,1)\\
	(0,0,1,1)\rightarrow(1,0,1,1);\
	(0,1,1,1)\rightarrow(1,1,1,1)\\
	(0,0,1,1)\rightarrow(0,1,1,1);\
	(0,1,1,1)\rightarrow(1,1,1,1)\\
	(0,0,0,1)\rightarrow(0,0,1,1);\
	(0,1,0,1)\rightarrow(0,1,1,1)\\
	(1,0,0,1)\rightarrow(1,0,1,1);\
	(1,1,0,1)\rightarrow(1,1,1,1)\\
	\\
	(0,0,0,0)\rightarrow(0,0,0,1);\
	(1,0,0,0)\rightarrow(1,0,0,1)\\
	(0,1,0,0)\rightarrow(0,1,0,1);\
	(1,1,0,0)\rightarrow(1,1,0,1)\\
	(0,0,1,0)\rightarrow(0,0,1,1);\
	(1,0,1,0)\rightarrow(1,0,1,1)\\
	(0,1,1,0)\rightarrow(0,1,1,1);\
	(1,1,1,0)\rightarrow(1,1,1,1)
}{hyperw}

Es ist leider nicht ohne weiteres Möglich, den Hyperwürfel zu zeichnen.
Ich möchte jedoch einen Versuch unternehmen, den vierdiemensionalen Würfel
auf drei Dimensionen zu projizieren, damit eine Visualisation möglich wird.
\end{document}